\documentclass{htblagkr}

\title{Titel der\\ Diplomarbeit}
\author{Max Mustermann}{5BHIF-12}
\author{Mein Name}{5BHIF-34}[\includegraphics{example-signature}]
\date{01.01.2023}
\location{Grieskirchen}

\discipline{Informatik}
\schoolyear{2022/2023}
\supervisor{Prof. Dr. Dr. Mustermann}

\begin{document}
    \maketitle

    \affidavit

    \section{Individualarbeitsrecht}
    Da die Interessen der Arbeitgeber und Arbeitnehmer von Natur aus sehr unterschiedlich sind, ist es wichtig, klare Vereinbarungen zu treffen, die auch bei späteren Meinungsverschiedenheiten nachgewiesen werden können.

    \subsection{Das Arbeitsverhältnis}
    Wenn ein Arbeitnehmer seine Arbeitskraft einem Arbeitgeber zur Verfügung stellt und dabei unter dessen Leitung in persönlicher und wirtschaftlicher Abhängigkeit tätig ist, so ist ein Arbeitsverhältnis entstanden.

    Grundsätzlich gilt auch im Arbeitsrecht der Stufenbau der Rechtsordnung.
    Die höherrangige Norm geht in der Regel der niedrigeren vor.
    Ist jedoch die niedrigere Norm für den Arbeitnehmer günstiger, so gilt diese.
    Außerdem wird der Stufenbau durch spezielle arbeitsrechtliche Normen ergänzt

    \subsubsection{Arten von Arbeitsverhältnissen}

    \paragraph{Unterscheidung nach Art der Verwendung}

    \subparagraph{Angestellter (Gehalt)}

    Darunter fallen alle Arbeitnehmer, welche

    \begin{itemize}
        \item{kaufmännische}
        \item{höhere nicht kaufmännische oder}
        \item{Kanzleiarbeiten erledigen}
    \end{itemize}

    \textbf{gesetzliche Grundlage:} Angestelltengesetz (AngG)

    \textbf{Beispiele:} Buchhalterin, Sekretär

    \subparagraph{Arbeiter (Lohn)}

    sind all jene Arbeitnehmer, welche nicht Tätigkeiten nach dem Angestelltengesetz verrichten und keine Lehrlinge sind.
    Die Erbringung von manuellen Tätigkeiten steht im Vordergrund.

    \textbf{gesetzliche Grundlagen:} Gewerbeordnung 1858, ABGB oder Sondergesetze

    \textbf{Beispiele:} Kfz-Mechaniker, Bäcker

\end{document}
